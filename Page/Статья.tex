\textheight = 478pt \textwidth=4.5in \topmargin=0.75in \headsep=0.25in
\topskip=10pt \footskip=0.35in
\documentclass[14pt, a4paper]{extbook}
\usepackage[T2A]{fontenc}
\usepackage[utf8]{inputenc}
\usepackage[russian]{babel}
\usepackage{mathtext,graphicx,indentfirst}
\usepackage{amsfonts,amssymb,amscd,amsmath,amsthm,wrapfig}
\usepackage[body={6.6in, 9.2in,left=1in, top=1.2in}]{geometry}
\usepackage{titlesec}
\usepackage{graphicx}
\usepackage{subcaption}

\titleformat*{\section}{\large\centering}
\captionsetup[sub]{labelformat=empty}


\begin{document}
{\sl УДК}\\ % Номер УДК
\medskip

%\begin{flushleft} 
%\copyright~Иванов~И.\,И., 2025 
%\end{flushleft}  % Авторские права, понадобится на этапе верстки

%\smallskip \noindent  %%Оформление теорем, лемм, замечаний и определений:
%{\bf Теорема 5.} {\sl Пусть при $h < 1$ одномерное отображение (3) имеет
%экспоненциально устойчивый цикл (4). Тогда существует такое $\varepsilon_k$,
%что при $ 0 < \varepsilon \leq \varepsilon_k $ уравнение (2) имеет
%релаксационный цикл.}
%\smallskip
%
%\noindent{\it Доказательство} %% ...
%
%\bigskip
%{\large\bf Литература} %%Список литературы
%
%\begin{enumerate}
%\item {\it Боголюбов Н.Н., Митропольский Ю.А.}
%Асимптотические методы в теории нелинейных колебаний. М.: Наука, 1974. %% ...
%\end{enumerate}

\begin{center}
{\bf О.\,Д.~Еремичев}\\[0.2cm] %% Автор
{\Large Устойчивость нулевого состояния равновесия в модельной системе ОДУ} % Название
\end{center}

\begin{quotation} \small \centering
Строится конечная нормальная форма системы интегрально-дифференциальных уравнений и по ней исследуется устойчивость нулевого состояния равновесия 
\end{quotation}

\section*{Постановка задачи} 
Рассматривается система интегрально-дифференциальных уравнений
\begin{equation}\label{eq:1}
\dfrac{du}{dt} = \left(A_0 + \varepsilon A_1\right)u + F_2(u,u) + F_3(u,u,u) + \varepsilon D_0 \dfrac{1}{2\pi} \int_{0}^{2\pi} u(t, x+s) \, ds
\end{equation}

Здесь \( u = u(t, x) \in \mathbb{R}^2 \), \( t \geq 0 \), \( x \in \mathbb{R} \), \(\varepsilon\) -- малый положительный параметр,  
\( A_0, A_1, D_0 \) -- \( 2 \times 2 \) матрицы, \( F_2(\cdot, \cdot), F_3(\cdot, \cdot, \cdot) \) -- линейные по каждому аргументу вектор-функции.

Система (1) рассматривается с периодическим краевым условием

\begin{equation}\label{eq:2}
u(t, x + 2\pi) = u(t, x).
\end{equation}

Рассмотрим задачу определения устойчивости нулевого состояния равновесия. Для этого воспользуемся следующей теоремой.

\smallskip \noindent
{\bf Теорема об устойчивости по первому приближению.} {\sl 

$\dot{x}=f(x,\varepsilon)$, $x=x(t)\in \mathbb{R}^n$

$\dot{x_0}=0$

$f(0,\varepsilon)=0$, $x=0$ - состояние равновесия

$f(x,\varepsilon)=f(0,\varepsilon)+\dfrac{df}{dt}(0,\varepsilon)x+...$

$A(\varepsilon)=\dfrac{df}{dt}(0,\varepsilon)$

$\dot{x}=A(\varepsilon)x+...$

$A(\varepsilon): A(\varepsilon)\upsilon=\lambda\upsilon$

$Re\lambda<0$ $\to$ состояние равновесия устойчиво

$Re\lambda>0$ $\to$ состояние равновесия неустойчиво
}
\smallskip

Теперь необходимо воспользоваться методом нормальных форм

$\dot{x}=(A_0 + \varepsilon A_1)x+F_2(x,x)+F3(x,x,x)$

$x(t)=\varepsilon \xi(\tau)a+\varepsilon^2x_2(\tau,t)+...$, $\tau=\varepsilon t$

$A_0a=0$

$x_2(\tau,t)$ ограничена

$\varepsilon^1: 0=A_0(\xi a)$

$\varepsilon^2: \dfrac{dx_2}{dt}+\dfrac{d\xi_2}{d\tau}a=A_0x_2+A_1\xi a+F_2(a,a)\xi^2$

$\dot{y}=-A_0^*y$ - сопряженное уравнение

$A_0^*=\overline{A_0}^T$

$y(t)=Ce^{\lambda t}$

$y(t)=Cb$

$-A_0^*b=0: (a,b)=1$

$\left(\dfrac{d\xi}{d\tau}a,b\right)=(\xi A_1+\xi^2 F_2(a,a),b)$

$\dfrac{d\xi}{d\tau}=(A_1a,b)\xi+(F_2(a,a),b)\xi^2$

$\lambda=(A_1a,b)$

$\sigma=(F_2(a,a),b)$

$\dfrac{d\xi}{d\tau}=\lambda\xi+\sigma\xi^2$ - нормальная форма

$\lambda>0,\sigma<0 \to \xi_0=0$ - неустойчиво

$\xi_1=\dfrac{-\lambda}{\delta}$

$x(t)=\varepsilon\xi_i a + O(\varepsilon^2)$

Если $(F2(a,a),b)=0$

$x(t)=\sqrt{\varepsilon}\xi(\tau)a+\varepsilon x_2(t,\tau)+\varepsilon^{\frac{3}{2}}x_3(t,\tau)+...$

$\varepsilon^2: \dfrac{dx_3}{dt}+\dfrac{d\xi}{d\tau}a=A_0x_3+K_0\xi^3$

$\dfrac{d\xi}{d\tau}=\lambda\xi+\delta_1\xi^3, \lambda>0,\delta<0$

$\lambda\xi+\delta_1\xi^3=0, \xi_0=0$ - неустойчивое состояние равновесия

$\xi_{1, 2}=\pm\sqrt{\dfrac{-\lambda}{\delta_1}}$ - устойчивое состояние равновесия

$A_0a=i\omega a$

$x(t)=\sqrt{\varepsilon}(\xi(\tau)e^{i\omega\tau}a+\overline{\xi}(\tau)e^{-i\omega\tau}\overline{a})+\varepsilon x_2(t,\tau)+\varepsilon^{\frac{3}{2}}x_3(t,\tau)+...$

$x2, x3$ - периодические по $t$ с $T=\dfrac{2\pi}{\omega}$

$\dfrac{dx_3}{dt}+\dfrac{d\xi}{d\tau}a=A_0x_3+...$

$\dfrac{d\xi}{d\tau}=\lambda\xi+\delta\xi|\xi|^2, \xi=\xi(\tau)\in\mathbb{C}, |\xi|^2=\xi\overline{\xi}$

$\xi_0=0$ - неустойчивое состояние равновесия

$\xi_{1, 2}\neq 0$ - устойчивое состояние равновесия

\section*{Построение конечной нормальной формы} 

Рассмотрим вопрос об устойчивости нулевого состояния равновесия задачи \eqref{eq:1}, \eqref{eq:2}.  Устойчивость нулевого состояния равновесия определяется собственными значениями набора матриц

\begin{equation}\label{eq:3}
A_0 + \varepsilon A_1 + \varepsilon D_0 \frac{1}{2\pi} \int_{0}^{2\pi} \exp(iks) \, ds, 
\quad (k = 0, \pm 1, \pm 2, \ldots).
\end{equation}

Заметим, что в этом случае при \( k \neq 0 \) матрицы вида \eqref{eq:3} равны матрице \( A_0 + \varepsilon A_1 \). 
При \( k = 0 \) получим матрицу \( A_0 + \varepsilon (A_1 + D_0) \). Таким образом, устойчивость нулевого 
состояния равновесия задачи \eqref{eq:1}, \eqref{eq:2} определяется в главном собственными значениями матрицы \( A_0 \).

Заметим также, что при замене переменной $x = x + s$ в интеграле системы \eqref{eq:1} в самой задаче меняется лишь аргумент в подынтегральной функции и соответствующее интегральное слагаемое будет иметь вид $\varepsilon D_0 \frac{1}{2\pi} \int_{0}^{2\pi} u(t, x)  dx$.

Выражение \(\frac{1}{2\pi} \int_{0}^{2\pi} u(t, x)  dx\) является средним функции \(u\) по переменной \(x\) на отрезке \([0, 2\pi]\). В дальнейшем будем считать, что соответствующая замена переменных выполнена, и будем обозначать среднее функции \(u\) как \(M(u)\).

Рассмотрим следующий критический случай: матрица \( A_0 \) имеет пару чисто мнимых собственных значений, т. е. $\lambda=\pm i \omega_0, \omega_0>0$.

Существует собственный вектор $a$, удовлетворяющий равенству $A_0 a=i\omega a$. Введем также в рассмотрение собственный вектор \( b \) 
транспонированной к \( A_0 \) матрице, отвечающий сопряженному собственному значению, 
то есть удовлетворяющий равенству \( A_0^T b = -i\omega b \). Будем считать, что \((a,b) = 1\). 
Тогда решение задачи \eqref{eq:1}, \eqref{eq:2} будем искать в виде ряда
\[
u(t,x) = \varepsilon^{1/2} \left( \xi(\tau,x) a \exp(i\omega t) + \overline{\xi}(\tau,x) \overline{a} \exp(-i\omega t) \right) + \varepsilon u_2(t,\tau,x) + \varepsilon^{3/2} u_3(t,\tau,x) + \ldots,
\]

где \( u_2, u_3 \) -- периодические по \( t \) с периодом \( \frac{2\pi}{\omega} \), \( \tau = \varepsilon t \). 
Подставив этот ряд в систему \eqref{eq:1}, получим при \( \sqrt{\varepsilon} \) верное равенство
\[
\xi i\omega a \exp(i\omega \tau) - \overline{\xi} i\omega \overline{a} \exp(-i\omega \tau) = \xi A_0 a \exp(i\omega \tau) + \overline{\xi} A_0 \overline{a} \exp(-i\omega \tau).
\]

Проделав ряд математических преобразований получим следующую квазинормальную форму

\begin{equation}\label{eq:4}
\dfrac{\partial \xi}{\partial \tau} = \lambda\xi +\sigma\xi|\xi|^2 + \gamma M(\xi), \quad \xi(\tau, x + 2\pi) = \xi(\tau, x)
\end{equation}

где 
\[
\lambda=(A_1a,b), \gamma=(D_0a,b)
\]
\[
w_1 = (2i\omega I - A_0)^{-1} F_2(a, a), \quad w_2 = -A_0^{-1} \bigl( F_2(a, \bar{a}) + F_2(\bar{a}, a) \bigr).
\]
\[
\sigma = \bigl( F_3(\bar{a}, a, a) + F_3(a, \bar{a}, a) + F_3(a, a, \bar{a}) + F_2(w_2, a) + F_2(a, w_2) + F_2(w_1, \bar{a}) + F_2(\bar{a}, w_1), b \bigr).
\]

Будем считать, что $\lambda=\lambda_0+i\lambda_1, \sigma=\sigma_0+i\sigma_1, \gamma=\gamma_0+i\gamma_1$.

\section*{Кусочно-постоянное решение}

Будем искать кусочно-постоянное решение задачи \eqref{eq:4} в виде
\begin{equation}
\xi(\tau, x) = \rho(x) e^{i\omega\tau},
\end{equation}
где
\[
\rho(x) =
\begin{cases}
\rho_1 e^{i\varphi_1}, & x \in [0, \alpha), \\
\rho_2 e^{i\varphi_2}, & x \in [\alpha, 2\pi).
\end{cases}
\]

С помощью нормировок и переобозначений можно получить $\varphi_1 = 0$. Будем считать в этом случае $\varphi_2 = \varphi$.

Подставляя $\xi(\tau, x)$ в уравнение~(23) и выделяя вещественную и мнимую части, получаем систему алгебраических уравнений

\begin{equation}\label{eq:6}
\begin{aligned}
\beta \rho_1 = \lambda_1 \rho_1 + \sigma_1 \rho_1^3 
+ \gamma_1 \frac{1}{2\pi} \rho_1 \alpha 
+ \gamma_1 \rho_2 (2\pi - \alpha) \frac{1}{2\pi} \cos(\varphi) 
+ \rho_2 (2\pi - \alpha) \frac{1}{2\pi} \gamma_0 \sin(\varphi), \\
%
0 = \lambda_0 \rho_1 + \sigma_0 \rho_1^3 
+ \gamma_0 \frac{1}{2\pi} \rho_1 \alpha 
+ \rho_2 (2\pi - \alpha) \frac{1}{2\pi} \gamma_0 \cos(\varphi)
- \gamma_1 \rho_2 (2\pi - \alpha) \frac{1}{2\pi} \sin(\varphi), \\
%
\rho_2 \beta \cos(\varphi) = \lambda_0 \rho_2 \sin(\varphi)
+ \lambda_1 \rho_2 \cos(\varphi)
+ \sigma_0 \rho_2^3 \sin(\varphi)
+ \sigma_1 \rho_2^3 \cos(\varphi)
+ \frac{1}{2\pi} \gamma_1 \rho_1 \alpha \\
+ \frac{1}{2\pi} \gamma_0 \rho_2 (2\pi - \alpha) \sin(\varphi)
 + \frac{1}{2\pi} \gamma_1 (2\pi - \alpha) \rho_2 \cos(\varphi)\\
 %
 -\rho_2 \beta \sin(\varphi) = \lambda_0 \rho_2 \cos(\varphi)
- \lambda_1 \rho_2 \sin(\varphi)
+ \sigma_0 \rho_2^3 \cos(\varphi)
- \sigma_1 \rho_2^3 \sin(\varphi)
+ \frac{1}{2\pi} \gamma_0 \rho_1 \alpha \\
+ \frac{1}{2\pi} \gamma_0 \rho_2 (2\pi - \alpha) \cos(\varphi)
- \frac{1}{2\pi} \gamma_1 (2\pi - \alpha) \rho_2 \sin(\varphi).
\end{aligned}
\end{equation}

\section*{Примеры $Re \xi(\tau, x)$}

Для раcчета коэффициентов нормальной формы $\lambda, \sigma, \gamma$ и решения системы алгебраических уравнений \eqref{eq:6} относительно неизвестных $\rho_1, \rho_2, \alpha, \beta$ при $\varphi \in [0, 2\pi]$ были написаны скрипты на Wolfram Mathematica.

%$\lambda=i, \sigma=i, \gamma=-1, \varphi=0, \rho_1=-1, \rho_2=1, \alpha=\pi, \beta=2, \tau=0$

\begin{figure}[ht]
    \centering
    \begin{subfigure}{0.45\textwidth}
        \centering
        \includegraphics[width=\linewidth]{image2.eps}
        \caption{$\tau=0$}
        \label{fig1:sub1}
    \end{subfigure}
    \hfill
    \begin{subfigure}{0.45\textwidth}
        \centering
        \includegraphics[width=\linewidth]{image3.eps}
        \caption{$\tau=100$}
        \label{fig1:sub2}
    \end{subfigure}
    
    \caption{$\lambda=1, \sigma=-2+18i, \gamma=0, \varphi=\dfrac{\pi}{2}, \rho_1=\dfrac{1}{\sqrt{2}}, \rho_2=-\dfrac{1}{\sqrt{2}}, \alpha=1, \beta=9$}
    \label{fig1:grid}
\end{figure}

\begin{figure}[ht]
    \centering
    \begin{subfigure}{0.45\textwidth}
        \centering
        \includegraphics[width=\linewidth]{image1.eps}
        \caption{$\tau=0$}
        \label{fig2:sub1}
    \end{subfigure}
    \hfill
    \begin{subfigure}{0.45\textwidth}
        \centering
        \includegraphics[width=\linewidth]{image4.eps}
        \caption{$\tau=100$}
        \label{fig2:sub2}
    \end{subfigure}
    
    \caption{$\lambda=i, \sigma=i, \gamma=-1, \varphi=0, \rho_1=-1, \rho_2=1, \alpha=\pi, \beta=2, \tau=0$}
    \label{fig2:grid}
\end{figure}
\begin{flushright}
\vfill
{\it Ярославский государственный университет им. П.Г. Демидова}
\end{flushright}

\end{document}
